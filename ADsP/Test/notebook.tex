
% Default to the notebook output style

    


% Inherit from the specified cell style.




    
\documentclass[11pt]{article}

    
    
    \usepackage[T1]{fontenc}
    % Nicer default font (+ math font) than Computer Modern for most use cases
    \usepackage{mathpazo}

    % Basic figure setup, for now with no caption control since it's done
    % automatically by Pandoc (which extracts ![](path) syntax from Markdown).
    \usepackage{graphicx}
    % We will generate all images so they have a width \maxwidth. This means
    % that they will get their normal width if they fit onto the page, but
    % are scaled down if they would overflow the margins.
    \makeatletter
    \def\maxwidth{\ifdim\Gin@nat@width>\linewidth\linewidth
    \else\Gin@nat@width\fi}
    \makeatother
    \let\Oldincludegraphics\includegraphics
    % Set max figure width to be 80% of text width, for now hardcoded.
    \renewcommand{\includegraphics}[1]{\Oldincludegraphics[width=.8\maxwidth]{#1}}
    % Ensure that by default, figures have no caption (until we provide a
    % proper Figure object with a Caption API and a way to capture that
    % in the conversion process - todo).
    \usepackage{caption}
    \DeclareCaptionLabelFormat{nolabel}{}
    \captionsetup{labelformat=nolabel}

    \usepackage{adjustbox} % Used to constrain images to a maximum size 
    \usepackage{xcolor} % Allow colors to be defined
    \usepackage{enumerate} % Needed for markdown enumerations to work
    \usepackage{geometry} % Used to adjust the document margins
    \usepackage{amsmath} % Equations
    \usepackage{amssymb} % Equations
    \usepackage{textcomp} % defines textquotesingle
    % Hack from http://tex.stackexchange.com/a/47451/13684:
    \AtBeginDocument{%
        \def\PYZsq{\textquotesingle}% Upright quotes in Pygmentized code
    }
    \usepackage{upquote} % Upright quotes for verbatim code
    \usepackage{eurosym} % defines \euro
    \usepackage[mathletters]{ucs} % Extended unicode (utf-8) support
    \usepackage[utf8x]{inputenc} % Allow utf-8 characters in the tex document
    \usepackage{fancyvrb} % verbatim replacement that allows latex
    \usepackage{grffile} % extends the file name processing of package graphics 
                         % to support a larger range 
    % The hyperref package gives us a pdf with properly built
    % internal navigation ('pdf bookmarks' for the table of contents,
    % internal cross-reference links, web links for URLs, etc.)
    \usepackage{hyperref}
    \usepackage{longtable} % longtable support required by pandoc >1.10
    \usepackage{booktabs}  % table support for pandoc > 1.12.2
    \usepackage[inline]{enumitem} % IRkernel/repr support (it uses the enumerate* environment)
    \usepackage[normalem]{ulem} % ulem is needed to support strikethroughs (\sout)
                                % normalem makes italics be italics, not underlines
    

    
    
    % Colors for the hyperref package
    \definecolor{urlcolor}{rgb}{0,.145,.698}
    \definecolor{linkcolor}{rgb}{.71,0.21,0.01}
    \definecolor{citecolor}{rgb}{.12,.54,.11}

    % ANSI colors
    \definecolor{ansi-black}{HTML}{3E424D}
    \definecolor{ansi-black-intense}{HTML}{282C36}
    \definecolor{ansi-red}{HTML}{E75C58}
    \definecolor{ansi-red-intense}{HTML}{B22B31}
    \definecolor{ansi-green}{HTML}{00A250}
    \definecolor{ansi-green-intense}{HTML}{007427}
    \definecolor{ansi-yellow}{HTML}{DDB62B}
    \definecolor{ansi-yellow-intense}{HTML}{B27D12}
    \definecolor{ansi-blue}{HTML}{208FFB}
    \definecolor{ansi-blue-intense}{HTML}{0065CA}
    \definecolor{ansi-magenta}{HTML}{D160C4}
    \definecolor{ansi-magenta-intense}{HTML}{A03196}
    \definecolor{ansi-cyan}{HTML}{60C6C8}
    \definecolor{ansi-cyan-intense}{HTML}{258F8F}
    \definecolor{ansi-white}{HTML}{C5C1B4}
    \definecolor{ansi-white-intense}{HTML}{A1A6B2}

    % commands and environments needed by pandoc snippets
    % extracted from the output of `pandoc -s`
    \providecommand{\tightlist}{%
      \setlength{\itemsep}{0pt}\setlength{\parskip}{0pt}}
    \DefineVerbatimEnvironment{Highlighting}{Verbatim}{commandchars=\\\{\}}
    % Add ',fontsize=\small' for more characters per line
    \newenvironment{Shaded}{}{}
    \newcommand{\KeywordTok}[1]{\textcolor[rgb]{0.00,0.44,0.13}{\textbf{{#1}}}}
    \newcommand{\DataTypeTok}[1]{\textcolor[rgb]{0.56,0.13,0.00}{{#1}}}
    \newcommand{\DecValTok}[1]{\textcolor[rgb]{0.25,0.63,0.44}{{#1}}}
    \newcommand{\BaseNTok}[1]{\textcolor[rgb]{0.25,0.63,0.44}{{#1}}}
    \newcommand{\FloatTok}[1]{\textcolor[rgb]{0.25,0.63,0.44}{{#1}}}
    \newcommand{\CharTok}[1]{\textcolor[rgb]{0.25,0.44,0.63}{{#1}}}
    \newcommand{\StringTok}[1]{\textcolor[rgb]{0.25,0.44,0.63}{{#1}}}
    \newcommand{\CommentTok}[1]{\textcolor[rgb]{0.38,0.63,0.69}{\textit{{#1}}}}
    \newcommand{\OtherTok}[1]{\textcolor[rgb]{0.00,0.44,0.13}{{#1}}}
    \newcommand{\AlertTok}[1]{\textcolor[rgb]{1.00,0.00,0.00}{\textbf{{#1}}}}
    \newcommand{\FunctionTok}[1]{\textcolor[rgb]{0.02,0.16,0.49}{{#1}}}
    \newcommand{\RegionMarkerTok}[1]{{#1}}
    \newcommand{\ErrorTok}[1]{\textcolor[rgb]{1.00,0.00,0.00}{\textbf{{#1}}}}
    \newcommand{\NormalTok}[1]{{#1}}
    
    % Additional commands for more recent versions of Pandoc
    \newcommand{\ConstantTok}[1]{\textcolor[rgb]{0.53,0.00,0.00}{{#1}}}
    \newcommand{\SpecialCharTok}[1]{\textcolor[rgb]{0.25,0.44,0.63}{{#1}}}
    \newcommand{\VerbatimStringTok}[1]{\textcolor[rgb]{0.25,0.44,0.63}{{#1}}}
    \newcommand{\SpecialStringTok}[1]{\textcolor[rgb]{0.73,0.40,0.53}{{#1}}}
    \newcommand{\ImportTok}[1]{{#1}}
    \newcommand{\DocumentationTok}[1]{\textcolor[rgb]{0.73,0.13,0.13}{\textit{{#1}}}}
    \newcommand{\AnnotationTok}[1]{\textcolor[rgb]{0.38,0.63,0.69}{\textbf{\textit{{#1}}}}}
    \newcommand{\CommentVarTok}[1]{\textcolor[rgb]{0.38,0.63,0.69}{\textbf{\textit{{#1}}}}}
    \newcommand{\VariableTok}[1]{\textcolor[rgb]{0.10,0.09,0.49}{{#1}}}
    \newcommand{\ControlFlowTok}[1]{\textcolor[rgb]{0.00,0.44,0.13}{\textbf{{#1}}}}
    \newcommand{\OperatorTok}[1]{\textcolor[rgb]{0.40,0.40,0.40}{{#1}}}
    \newcommand{\BuiltInTok}[1]{{#1}}
    \newcommand{\ExtensionTok}[1]{{#1}}
    \newcommand{\PreprocessorTok}[1]{\textcolor[rgb]{0.74,0.48,0.00}{{#1}}}
    \newcommand{\AttributeTok}[1]{\textcolor[rgb]{0.49,0.56,0.16}{{#1}}}
    \newcommand{\InformationTok}[1]{\textcolor[rgb]{0.38,0.63,0.69}{\textbf{\textit{{#1}}}}}
    \newcommand{\WarningTok}[1]{\textcolor[rgb]{0.38,0.63,0.69}{\textbf{\textit{{#1}}}}}
    
    
    % Define a nice break command that doesn't care if a line doesn't already
    % exist.
    \def\br{\hspace*{\fill} \\* }
    % Math Jax compatability definitions
    \def\gt{>}
    \def\lt{<}
    % Document parameters
    \title{[ADSP] BlankTest\_05\_Statistics}
    
    
    

    % Pygments definitions
    
\makeatletter
\def\PY@reset{\let\PY@it=\relax \let\PY@bf=\relax%
    \let\PY@ul=\relax \let\PY@tc=\relax%
    \let\PY@bc=\relax \let\PY@ff=\relax}
\def\PY@tok#1{\csname PY@tok@#1\endcsname}
\def\PY@toks#1+{\ifx\relax#1\empty\else%
    \PY@tok{#1}\expandafter\PY@toks\fi}
\def\PY@do#1{\PY@bc{\PY@tc{\PY@ul{%
    \PY@it{\PY@bf{\PY@ff{#1}}}}}}}
\def\PY#1#2{\PY@reset\PY@toks#1+\relax+\PY@do{#2}}

\expandafter\def\csname PY@tok@w\endcsname{\def\PY@tc##1{\textcolor[rgb]{0.73,0.73,0.73}{##1}}}
\expandafter\def\csname PY@tok@c\endcsname{\let\PY@it=\textit\def\PY@tc##1{\textcolor[rgb]{0.25,0.50,0.50}{##1}}}
\expandafter\def\csname PY@tok@cp\endcsname{\def\PY@tc##1{\textcolor[rgb]{0.74,0.48,0.00}{##1}}}
\expandafter\def\csname PY@tok@k\endcsname{\let\PY@bf=\textbf\def\PY@tc##1{\textcolor[rgb]{0.00,0.50,0.00}{##1}}}
\expandafter\def\csname PY@tok@kp\endcsname{\def\PY@tc##1{\textcolor[rgb]{0.00,0.50,0.00}{##1}}}
\expandafter\def\csname PY@tok@kt\endcsname{\def\PY@tc##1{\textcolor[rgb]{0.69,0.00,0.25}{##1}}}
\expandafter\def\csname PY@tok@o\endcsname{\def\PY@tc##1{\textcolor[rgb]{0.40,0.40,0.40}{##1}}}
\expandafter\def\csname PY@tok@ow\endcsname{\let\PY@bf=\textbf\def\PY@tc##1{\textcolor[rgb]{0.67,0.13,1.00}{##1}}}
\expandafter\def\csname PY@tok@nb\endcsname{\def\PY@tc##1{\textcolor[rgb]{0.00,0.50,0.00}{##1}}}
\expandafter\def\csname PY@tok@nf\endcsname{\def\PY@tc##1{\textcolor[rgb]{0.00,0.00,1.00}{##1}}}
\expandafter\def\csname PY@tok@nc\endcsname{\let\PY@bf=\textbf\def\PY@tc##1{\textcolor[rgb]{0.00,0.00,1.00}{##1}}}
\expandafter\def\csname PY@tok@nn\endcsname{\let\PY@bf=\textbf\def\PY@tc##1{\textcolor[rgb]{0.00,0.00,1.00}{##1}}}
\expandafter\def\csname PY@tok@ne\endcsname{\let\PY@bf=\textbf\def\PY@tc##1{\textcolor[rgb]{0.82,0.25,0.23}{##1}}}
\expandafter\def\csname PY@tok@nv\endcsname{\def\PY@tc##1{\textcolor[rgb]{0.10,0.09,0.49}{##1}}}
\expandafter\def\csname PY@tok@no\endcsname{\def\PY@tc##1{\textcolor[rgb]{0.53,0.00,0.00}{##1}}}
\expandafter\def\csname PY@tok@nl\endcsname{\def\PY@tc##1{\textcolor[rgb]{0.63,0.63,0.00}{##1}}}
\expandafter\def\csname PY@tok@ni\endcsname{\let\PY@bf=\textbf\def\PY@tc##1{\textcolor[rgb]{0.60,0.60,0.60}{##1}}}
\expandafter\def\csname PY@tok@na\endcsname{\def\PY@tc##1{\textcolor[rgb]{0.49,0.56,0.16}{##1}}}
\expandafter\def\csname PY@tok@nt\endcsname{\let\PY@bf=\textbf\def\PY@tc##1{\textcolor[rgb]{0.00,0.50,0.00}{##1}}}
\expandafter\def\csname PY@tok@nd\endcsname{\def\PY@tc##1{\textcolor[rgb]{0.67,0.13,1.00}{##1}}}
\expandafter\def\csname PY@tok@s\endcsname{\def\PY@tc##1{\textcolor[rgb]{0.73,0.13,0.13}{##1}}}
\expandafter\def\csname PY@tok@sd\endcsname{\let\PY@it=\textit\def\PY@tc##1{\textcolor[rgb]{0.73,0.13,0.13}{##1}}}
\expandafter\def\csname PY@tok@si\endcsname{\let\PY@bf=\textbf\def\PY@tc##1{\textcolor[rgb]{0.73,0.40,0.53}{##1}}}
\expandafter\def\csname PY@tok@se\endcsname{\let\PY@bf=\textbf\def\PY@tc##1{\textcolor[rgb]{0.73,0.40,0.13}{##1}}}
\expandafter\def\csname PY@tok@sr\endcsname{\def\PY@tc##1{\textcolor[rgb]{0.73,0.40,0.53}{##1}}}
\expandafter\def\csname PY@tok@ss\endcsname{\def\PY@tc##1{\textcolor[rgb]{0.10,0.09,0.49}{##1}}}
\expandafter\def\csname PY@tok@sx\endcsname{\def\PY@tc##1{\textcolor[rgb]{0.00,0.50,0.00}{##1}}}
\expandafter\def\csname PY@tok@m\endcsname{\def\PY@tc##1{\textcolor[rgb]{0.40,0.40,0.40}{##1}}}
\expandafter\def\csname PY@tok@gh\endcsname{\let\PY@bf=\textbf\def\PY@tc##1{\textcolor[rgb]{0.00,0.00,0.50}{##1}}}
\expandafter\def\csname PY@tok@gu\endcsname{\let\PY@bf=\textbf\def\PY@tc##1{\textcolor[rgb]{0.50,0.00,0.50}{##1}}}
\expandafter\def\csname PY@tok@gd\endcsname{\def\PY@tc##1{\textcolor[rgb]{0.63,0.00,0.00}{##1}}}
\expandafter\def\csname PY@tok@gi\endcsname{\def\PY@tc##1{\textcolor[rgb]{0.00,0.63,0.00}{##1}}}
\expandafter\def\csname PY@tok@gr\endcsname{\def\PY@tc##1{\textcolor[rgb]{1.00,0.00,0.00}{##1}}}
\expandafter\def\csname PY@tok@ge\endcsname{\let\PY@it=\textit}
\expandafter\def\csname PY@tok@gs\endcsname{\let\PY@bf=\textbf}
\expandafter\def\csname PY@tok@gp\endcsname{\let\PY@bf=\textbf\def\PY@tc##1{\textcolor[rgb]{0.00,0.00,0.50}{##1}}}
\expandafter\def\csname PY@tok@go\endcsname{\def\PY@tc##1{\textcolor[rgb]{0.53,0.53,0.53}{##1}}}
\expandafter\def\csname PY@tok@gt\endcsname{\def\PY@tc##1{\textcolor[rgb]{0.00,0.27,0.87}{##1}}}
\expandafter\def\csname PY@tok@err\endcsname{\def\PY@bc##1{\setlength{\fboxsep}{0pt}\fcolorbox[rgb]{1.00,0.00,0.00}{1,1,1}{\strut ##1}}}
\expandafter\def\csname PY@tok@kc\endcsname{\let\PY@bf=\textbf\def\PY@tc##1{\textcolor[rgb]{0.00,0.50,0.00}{##1}}}
\expandafter\def\csname PY@tok@kd\endcsname{\let\PY@bf=\textbf\def\PY@tc##1{\textcolor[rgb]{0.00,0.50,0.00}{##1}}}
\expandafter\def\csname PY@tok@kn\endcsname{\let\PY@bf=\textbf\def\PY@tc##1{\textcolor[rgb]{0.00,0.50,0.00}{##1}}}
\expandafter\def\csname PY@tok@kr\endcsname{\let\PY@bf=\textbf\def\PY@tc##1{\textcolor[rgb]{0.00,0.50,0.00}{##1}}}
\expandafter\def\csname PY@tok@bp\endcsname{\def\PY@tc##1{\textcolor[rgb]{0.00,0.50,0.00}{##1}}}
\expandafter\def\csname PY@tok@fm\endcsname{\def\PY@tc##1{\textcolor[rgb]{0.00,0.00,1.00}{##1}}}
\expandafter\def\csname PY@tok@vc\endcsname{\def\PY@tc##1{\textcolor[rgb]{0.10,0.09,0.49}{##1}}}
\expandafter\def\csname PY@tok@vg\endcsname{\def\PY@tc##1{\textcolor[rgb]{0.10,0.09,0.49}{##1}}}
\expandafter\def\csname PY@tok@vi\endcsname{\def\PY@tc##1{\textcolor[rgb]{0.10,0.09,0.49}{##1}}}
\expandafter\def\csname PY@tok@vm\endcsname{\def\PY@tc##1{\textcolor[rgb]{0.10,0.09,0.49}{##1}}}
\expandafter\def\csname PY@tok@sa\endcsname{\def\PY@tc##1{\textcolor[rgb]{0.73,0.13,0.13}{##1}}}
\expandafter\def\csname PY@tok@sb\endcsname{\def\PY@tc##1{\textcolor[rgb]{0.73,0.13,0.13}{##1}}}
\expandafter\def\csname PY@tok@sc\endcsname{\def\PY@tc##1{\textcolor[rgb]{0.73,0.13,0.13}{##1}}}
\expandafter\def\csname PY@tok@dl\endcsname{\def\PY@tc##1{\textcolor[rgb]{0.73,0.13,0.13}{##1}}}
\expandafter\def\csname PY@tok@s2\endcsname{\def\PY@tc##1{\textcolor[rgb]{0.73,0.13,0.13}{##1}}}
\expandafter\def\csname PY@tok@sh\endcsname{\def\PY@tc##1{\textcolor[rgb]{0.73,0.13,0.13}{##1}}}
\expandafter\def\csname PY@tok@s1\endcsname{\def\PY@tc##1{\textcolor[rgb]{0.73,0.13,0.13}{##1}}}
\expandafter\def\csname PY@tok@mb\endcsname{\def\PY@tc##1{\textcolor[rgb]{0.40,0.40,0.40}{##1}}}
\expandafter\def\csname PY@tok@mf\endcsname{\def\PY@tc##1{\textcolor[rgb]{0.40,0.40,0.40}{##1}}}
\expandafter\def\csname PY@tok@mh\endcsname{\def\PY@tc##1{\textcolor[rgb]{0.40,0.40,0.40}{##1}}}
\expandafter\def\csname PY@tok@mi\endcsname{\def\PY@tc##1{\textcolor[rgb]{0.40,0.40,0.40}{##1}}}
\expandafter\def\csname PY@tok@il\endcsname{\def\PY@tc##1{\textcolor[rgb]{0.40,0.40,0.40}{##1}}}
\expandafter\def\csname PY@tok@mo\endcsname{\def\PY@tc##1{\textcolor[rgb]{0.40,0.40,0.40}{##1}}}
\expandafter\def\csname PY@tok@ch\endcsname{\let\PY@it=\textit\def\PY@tc##1{\textcolor[rgb]{0.25,0.50,0.50}{##1}}}
\expandafter\def\csname PY@tok@cm\endcsname{\let\PY@it=\textit\def\PY@tc##1{\textcolor[rgb]{0.25,0.50,0.50}{##1}}}
\expandafter\def\csname PY@tok@cpf\endcsname{\let\PY@it=\textit\def\PY@tc##1{\textcolor[rgb]{0.25,0.50,0.50}{##1}}}
\expandafter\def\csname PY@tok@c1\endcsname{\let\PY@it=\textit\def\PY@tc##1{\textcolor[rgb]{0.25,0.50,0.50}{##1}}}
\expandafter\def\csname PY@tok@cs\endcsname{\let\PY@it=\textit\def\PY@tc##1{\textcolor[rgb]{0.25,0.50,0.50}{##1}}}

\def\PYZbs{\char`\\}
\def\PYZus{\char`\_}
\def\PYZob{\char`\{}
\def\PYZcb{\char`\}}
\def\PYZca{\char`\^}
\def\PYZam{\char`\&}
\def\PYZlt{\char`\<}
\def\PYZgt{\char`\>}
\def\PYZsh{\char`\#}
\def\PYZpc{\char`\%}
\def\PYZdl{\char`\$}
\def\PYZhy{\char`\-}
\def\PYZsq{\char`\'}
\def\PYZdq{\char`\"}
\def\PYZti{\char`\~}
% for compatibility with earlier versions
\def\PYZat{@}
\def\PYZlb{[}
\def\PYZrb{]}
\makeatother


    % Exact colors from NB
    \definecolor{incolor}{rgb}{0.0, 0.0, 0.5}
    \definecolor{outcolor}{rgb}{0.545, 0.0, 0.0}



    
    % Prevent overflowing lines due to hard-to-break entities
    \sloppy 
    % Setup hyperref package
    \hypersetup{
      breaklinks=true,  % so long urls are correctly broken across lines
      colorlinks=true,
      urlcolor=urlcolor,
      linkcolor=linkcolor,
      citecolor=citecolor,
      }
    % Slightly bigger margins than the latex defaults
    
    \geometry{verbose,tmargin=1in,bmargin=1in,lmargin=1in,rmargin=1in}
    
    

    \begin{document}
    
    
    \maketitle
    
    

    
    \begin{center}\rule{0.5\linewidth}{\linethickness}\end{center}

    \hypertarget{uxc81c4uxc7a5.-uxd1b5uxacc4uxbd84uxc11d}{%
\subsection{제4장.
통계분석}\label{uxc81c4uxc7a5.-uxd1b5uxacc4uxbd84uxc11d}}

    \hypertarget{uxd1b5uxacc4-uxbd84uxc11duxc758-uxc774uxd574}{%
\subsubsection{1. 통계 분석의
이해}\label{uxd1b5uxacc4-uxbd84uxc11duxc758-uxc774uxd574}}

\begin{enumerate}
\def\labelenumi{\arabic{enumi}.}
\tightlist
\item
  \textbf{통계}

  \begin{itemize}
  \tightlist
  \item
    ____ : 특정집단을 대상으로 수행한 ____나 ____ 통해 나온
    결과에 대한 요약된 형태의 표현
  \item
    획득방법 : ____와 ______
  \item
    ____기법 : ________, __________,
    __________
  \item
    자료의 형태 : ______, ______, ______,
    ________ 
  \end{itemize}
\item
  \textbf{통계분석}

  \begin{itemize}
  \tightlist
  \item
    ________ : 추설, 가설검정, 예측
  \item
    ________ : 평균, 표준편차, 중위수, 최빈값,, 그래프의 표현 
  \end{itemize}
\item
  \textbf{확률 및 확률 분포}

  \begin{itemize}
  \tightlist
  \item
    ______ : 특정값이 나타날 확률의 변수
  \item
    이산형 확률분포(________) : ______ 분포, ____
    분포, ____분포, ______분포, ______분포
  \item
    역속형 확률분포(________) : ______분포,
    ______분포, ______분포, __분포, __분포, __분포 
  \end{itemize}
\item
  \textbf{추정 및 가설 검정}

  \begin{itemize}
  \tightlist
  \item
    ____ : 표본으로 미지의 모수를 추측하는 것
  \item
    ______ : 모수가 특정 값으로 추정. ____, ______,
    ______ 등을 추정.

    \begin{itemize}
    \tightlist
    \item
      ______(________) 조건 : ____, ______,
      ____, ________
    \end{itemize}
  \item
    ______(________)

    \begin{itemize}
    \tightlist
    \item
      모수가 특정 구간에 있을 것으로 추정
    \item
      모분산을 알경우 ________ 활용, 모를경우 ________
      활용
    \end{itemize}
  \item
    가설검정

    \begin{itemize}
    \tightlist
    \item
      ________(________),
      ________(________)
    \item
      ____(____) : ____가설이 옳은데 귀묵가설을 기각하는
      오류
    \item
      ____(____) : ____가설이 옳지 않은데 ____을
      채택하는 오류
    \item
      ______ 크기를 0.1, 0.05, 0.01로 고정시키고 ______가
      최소가 되고록 ____을 설정 
    \end{itemize}
  \end{itemize}
\item
  \textbf{비모수 검정}

  \begin{itemize}
  \tightlist
  \item
    모집단의 분포에 아무 ____을 가하지 않고 실행하는 검정
  \item
    ``동일하다/동일하지 않다'' 식으로 가설 설정
  \item
    순위나 두 관측값 차이의 부호를 이용해 검정
  \item
    ________(________), ____의
    ________(________), ____의
    ________(________), ____의
    ________(________), ______, ______의
    ________(________)
  \end{itemize}
\end{enumerate}

    \hypertarget{uxae30uxc220-uxd1b5uxacc4-uxbd84uxc11d}{%
\subsubsection{2. 기술 통계
분석}\label{uxae30uxc220-uxd1b5uxacc4-uxbd84uxc11d}}

\begin{enumerate}
\def\labelenumi{\arabic{enumi}.}
\tightlist
\item
  \textbf{기술통계}

  \begin{itemize}
  \tightlist
  \item
    자료의 특성을 __, ____, ____ 등을 사용하여 쉽게 파악할 수
    있도록 정리/요약 
  \end{itemize}
\item
  \textbf{통계량에 의한 자료 분석}

  \begin{itemize}
  \tightlist
  \item
    ______ : 평균, 중앙값, 최빈값
  \item
    ____의 척도 : __, ____, ____, ________,
    변동계수, 표준오차
  \item
    분포의 형태 : ____, ____ 
  \end{itemize}
\item
  \textbf{그래프를 통한 자료 분석}

  \begin{itemize}
  \tightlist
  \item
    ____형 자료 : 막대그래프, 파이차트 등
  \item
    ____형 자료 : 히스토그램, 줄기-잎 그림, 상자그림
  \item
    ____ 자료 : 꺽은선 그래프 
  \end{itemize}
\item
  \textbf{연관성 분석}

  \begin{itemize}
  \tightlist
  \item
    ________, ________
  \item
    ________(________)로 확인할 수 있는 것

    \begin{itemize}
    \tightlist
    \item
      두 변수사의 선형관계, 함수관계 성립
    \item
      이상값의 존재 여부와 몇 개의 집단으로 구분
    \end{itemize}
  \item
    ________(________)

    \begin{itemize}
    \tightlist
    \item
      두 확률변수 간의 방향성을 확인 
    \end{itemize}
  \end{itemize}
\item
  \textbf{상관분석}

  \begin{itemize}
  \tightlist
  \item
    두 변수간의 ______정도를 ________를 통해 확인할 수
    있음
  \item
    -1에서 1사이의 값으로 표현, 0이면 ______가 없음
  \item
    ____________ : ____척도 이상으로 측정된 계수
  \item
    ____________ : ____ 및 ____ 척도로 측정된 계수
  \item
    프로그램

    \begin{itemize}
    \tightlist
    \item
      ______ : ______________
    \item
      ______ : ______________
    \end{itemize}
  \end{itemize}
\end{enumerate}

    \hypertarget{uxd68cuxadc0uxbd84uxc11d}{%
\subsubsection{3. 회귀분석}\label{uxd68cuxadc0uxbd84uxc11d}}

\begin{enumerate}
\def\labelenumi{\arabic{enumi}.}
\tightlist
\item
  \textbf{회귀분석}

  \begin{itemize}
  \tightlist
  \item
    ______ : 하나 이상의 ____변수들이 ____변수에 미치는
    영향을 추정할 수 있는 통계기법
  \item
    ____선형회귀 / ____선형회귀 
  \end{itemize}
\item
  \textbf{회귀분석 특징}

  \begin{itemize}
  \tightlist
  \item
    회귀식(모형)에 대한 검증 : ____검정
  \item
    회귀계수 검정 : ____ 검정
  \item
    모형의 설명력 : ____ 계수 = ______ /
    ______(____/____) 단 0과 1사이값
  \item
    선형회귀분석의 가정

    \begin{itemize}
    \tightlist
    \item
      ______ : 입력변수와 출력변수와 관계가 ____ : ____와
      출력변수의 ____로 확인
    \item
      ______ : 잔차와 ____ 값이 관련되어 있지 않음 :
      ____와 ____의 ____로 확인
    \item
      ______ : ______의 모든 값에 대한 오차들의 ____이
      일정
    \item
      ______ : 관측치들의 ____들이 상관이 없어야 함
    \item
      ______ : 잔차항이 ____분포를 이뤄야 함 
    \end{itemize}
  \end{itemize}
\item
  \textbf{다중선형회귀분석}

  \begin{itemize}
  \tightlist
  \item
    ______ : 변수들 사이에 ____관계가 존재하면 ______의
    정확한 추정이 어려움
  \item
    검사방법

    \begin{enumerate}
    \def\labelenumii{\arabic{enumii}.}
    \tightlist
    \item
      ________(______) : 10보다 크면 심각한 문제
    \item
      _______ : 10이상이면 문제, 30 보다 크면 심각
    \item
      ______가 강한 변수 제거 
    \end{enumerate}
  \end{itemize}
\item
  \textbf{변수선택법}

  \begin{itemize}
  \tightlist
  \item
    모든 가능한 조합 : 모든 가능한 변수의 조합에 대한 회귀모형을 분석
  \item
    ________(________) : 중요한 변수를 차례로 추가하는
    방법

    \begin{itemize}
    \tightlist
    \item
      이해 쉬운, 많은 변수 활용, 작은 변동에 결과가 달라져
      ______이 부족.
    \end{itemize}
  \item
    ________(________) : 독립변수 후보를 모두 포함한 후
    영향력이 적은 변수를 제거

    \begin{itemize}
    \tightlist
    \item
      전체 변수 정보 이용 가능, 변수가 많을 경우 활용 어려움,
      ______ 부족
    \end{itemize}
  \item
    ________(________) : 새롭게 추가한 변수의
    ______가 약화되면 제거
  \end{itemize}
\end{enumerate}

    \hypertarget{uxc2dcuxacc4uxc5f4-uxbd84uxc11d}{%
\subsubsection{4. 시계열 분석}\label{uxc2dcuxacc4uxc5f4-uxbd84uxc11d}}

\begin{enumerate}
\def\labelenumi{\arabic{enumi}.}
\tightlist
\item
  \textbf{시계열 자료}

  \begin{itemize}
  \tightlist
  \item
    ________ : 시간의 흐음에 따라 관찰된 값들
  \end{itemize}
\item
  \textbf{정상성}

  \begin{itemize}
  \tightlist
  \item
    3가지를 모두 만족 : ____이 일정 / ____이 일정 / ____
    특정시점에서 t,s에 의존하지 않고 일정
  \end{itemize}
\item
  \textbf{정상시계열의 특징}

  \begin{itemize}
  \tightlist
  \item
    어떤 시점에서 ____과 ____ 그리고 특정한 ____의 길이를
    갖는 ________을 측정하면 동일한 값
  \item
    항상 ______으로 회귀하려는 경향, ____은 평균값 주변에서
    일정한 __ 유지
  \item
    ____시계열은 특정 기간의 시계열 자료애서 얻은 정보를 다른 시기로
    ______ 할 수 없음
  \end{itemize}
\item
  \textbf{시계열 모형}

  \begin{itemize}
  \tightlist
  \item
    ________(________) : ACF는 빠르게 감소, PACF는
    절단점이 존재 : ____(절단점-1)
  \item
    ________(________) : ACF는 절단점이 존재, PACF는
    빠르게 감소
  \item
    ____________(__________)

    \begin{itemize}
    \tightlist
    \item
      d=0 이면 ______모형이라고 부르고, 정상성을 만족
    \item
      p=0 이면 ______모형이라고 부르고, d번 ____하면
      ____모형을 따름
    \end{itemize}
  \item
    __________

    \begin{itemize}
    \tightlist
    \item
      ________(________) : 형태가 오르거나 또는 내리는
      추세를 따르는 경우
    \item
      ________(________) : 요일, 월, 사분기 등 고정된
      주기에 따라 변화하는 경우
    \item
      ________(________) : 알려지지 않은 주기로 변화하는
      경우
    \item
      ________(________) : 이 세가지 요인으로 설명할 수
      없는 회귀분석에서 ____에 해당하는 요인
    \end{itemize}
  \end{itemize}
\end{enumerate}

    \hypertarget{uxb2e4uxcc28uxc6d0uxcc99uxb3c4uxbc95mds}{%
\subsubsection{5.
다차원척도법(MDS)}\label{uxb2e4uxcc28uxc6d0uxcc99uxb3c4uxbc95mds}}

\begin{enumerate}
\def\labelenumi{\arabic{enumi}.}
\item
  \textbf{다차원 척도법} 군집분석과 같이 개체들 사이의
  ____/______을 측정하여 2차원, 3차원 공간상에 점으로 표현하는
  방법

  \begin{itemize}
  \tightlist
  \item
    목적 : 군집분석은 개체들간의 동일한 그룹으로 분류 / 다차원척도법은
    점으로 표시하여 개체들 사이의 ____를 시각적으로 표현
  \item
    종류

    \begin{itemize}
    \tightlist
    \item
      ________(= ______________) : 구간이나
      비율척도일 경우, 각 개체들간의 ______ 거리를 계산
    \item
      ________ : 순서척도일 경우 활용. 순서척도일 경우 거리
      속성과 같도록 ______하여 거리르 생성
    \end{itemize}
  \item
    ______ 와 ______ 수준 M

    \begin{itemize}
    \tightlist
    \item
      개체들을 공간상에 표현하기 위한 방법
    \item
      ______ 나 ________를 부적합도 기준으로 사용
    \item
      부적합도를 최소로 하는 방법을 반복실행
    \end{itemize}
  \end{itemize}
\end{enumerate}

    \hypertarget{uxc8fcuxc131uxbd84uxbd84uxc11dpca}{%
\subsubsection{6.
주성분분석(PCA)}\label{uxc8fcuxc131uxbd84uxbd84uxc11dpca}}

\begin{enumerate}
\def\labelenumi{\arabic{enumi}.}
\tightlist
\item
  \textbf{주성분 분석}

  \begin{itemize}
  \item
    ____관계가 있는 변수들을 결합하여 ______관계가 없는 변수로
    분산을 극대화하는 방법
  \item
    변수를 ____하는데 사용
  \item
    ____분석 : 잠재된 변수를 추출하기 위한 작업
  \item
    ____분석 : 그 중 가장 많이 사용되는 방법
  \item
    공통점 : 데이터를 ____하는데 사용
  \item
    차이점

    \begin{itemize}
    \tightlist
    \item
      생성된 변수의 __와 ____ : 요인분석은 지정 가능, 주성분
      분석은 보통 2개
    \item
      변수와의 ____ : 요인분석은 ____한 관계, 주성분 분석은
      중요도에 따라 차이
    \item
      ____변수와의 관계 : 요인분석은 ____변수 고려안함, 주성분은
      ____변수를 고려하여 변수 생성
    \end{itemize}
  \item
  \end{itemize}
\item
  \textbf{주성분 분석의 활용}

  \begin{itemize}
  \tightlist
  \item
    여러 분석의 ______, ______을 이용해 주성분차원으로
    변수를 축소
  \item
    회귀나 의사결정나무 등에서 ________이 존재할 경우,
    ____가 높은 변수를 축소 
  \end{itemize}
\item
  \textbf{R결과 해석} \includegraphics{img/pca.png}

  \begin{itemize}
  \tightlist
  \item
    제 1주성분, 제 2주성분 누적 기여율은 ______ \%
  \item
    1 주성분의 기여울(해석률)은 ____\%, 2 주성분은 ____\%
  \end{itemize}
\end{enumerate}

    \begin{center}\rule{0.5\linewidth}{\linethickness}\end{center}


    % Add a bibliography block to the postdoc
    
    
    
    \end{document}
